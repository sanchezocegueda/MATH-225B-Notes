\setcounter{section}{0}
\counterwithout{section}{chapter}

Recall that a Turing machine $M$ is specified by the following parameters:
\begin{itemize}
    \item $Q$: a finite set of states, with two designated special states $q_{start}$ and $q_{halt}$.
    \item $\Sigma$ a finite alphabet.
    Usually, it includes the special `blank' symbol $\sqcup$.
    \item $t: Q \times \Sigma \to Q \times \Sigma \times \{L, R\}$, the transition function.
    Intuitively, $t$ decides what state to transition to next, what new symbol should go on the cell, and it also decides whether to move the head of the machine left or right.
\end{itemize}

A Turing machine $M$ computes the partial function $f : \Gamma\strings \to \Gamma\strings$ if $\Gamma \subseteq \Sigma$ and whenever $M$ is run starting with $s \in \Gamma\strings$ on the tape, tit halts with $t \in \Gamma\strings$ on the tape iff $f(s) = t$.

An important characteristic of Turing machines is that if $f: \Gamma\strings \to \Gamma \strings$ is partial computable, then it is computed by a Turing machine with alphabet $\Gamma$.

If we have an $f: \{0,1\}\strings \to \{0, 1\}\strings$, we can transform it to a function that works in unary.
For instance, some $g : \{1\}\strings \to \{1\}\strings$.
You can think of it as converting between bases.

\section{Basic Properties}
Below is a list of basic properties that Turing machines have
\begin{enumerate}
    \item There is a computable listing of all Turing machines $M_0, M_1, M_2, ...$ -- there are only countably many Turing Machines up to isomorphism.

    This means we can make a couple of simplifying assumptions:
    First, we can assume that whenever we have a Turing machine $M$ with $n$ states, then $Q_M = \{1, 2, ..., n\}$.
    Similarly, if the 


    The function $f(e) = n$ if $Q_{M_{e}} = \{1, ..., n\}$ is computable.
    The function $g(e) = n'$ if $\Sigma_{M_{e}} = \{1, ..., n'\}$ is computable ().
    The function $h(e, i, j) = t_{M_e}(i, j)$ is computable (transition function).

    An enumeration of all the Turing machines so that all its properties can be determined in a computable way.
    One can compute the description of the Turing machine from their index (by description we mean that states, the alphabet, and the transition function).
    
    \item Since we have this listing, we also have a listing of all the partial computable functions.
    $\varphi_e$ is the $e$th partial computable function, which is defined to be the function computed by $M_e$ (should say wrt what alphabet).
    So we can list them out as $\varphi_0, \varphi_1, \varphi_2, ...$.

    \item There exists a Universal Turing Machine.
    In other words, there is a partial computable function $\Phi:\nat \times \nat \to \nat N$ such that for any $e \in \nat$,
    $$\varphi(e, x) = \varphi_e(x)$$
    \textit{Proof?} We can show an algorithm.
    From $e$, it can compute the description of some Turing machine $M_e$, and then from that description it has computed, it computes $M_e(x) = \varphi_e(x)$.
    Note that each $\varphi_e : \nat \to \nat$.

    \item The second property is the Padding Lemma.
    The padding lemma says that for any partial computable function $\varphi : \nat \to \nat$, there are infinitely many $e \in \nat$ such that $\varphi = \varphi_e$.
    In other words, every partial computable function appears infinitely many times in our listing of partial computable functions.

    \textit{Proof?} We listed out all the Turing machines $M_0, M_1, M_2, ...$.
    Fix some $e_0$ such that $\varphi = \varphi_{e_0}$.
    So there is some $M_{e_0} = \varphi_{e_0} = \varphi$.
    But there are infinitely many Turing machines such that they are equal.
    One way we can do this is by adding more states while keeping the transition function the same.
    You can think about this as ``padding'' your Turing machine with extra states (hence the name of the lemma).
    Note that this is not necessarily a way to get all such functions, but it is enough to get infinitely many.

    Moreover, there is a computable function $f : \nat \times \nat \to \nat$ such that $f(e, \cdot) : \nat \to \nat$ is such that $\varphi_{f(e, k)} = \varphi_e$ for all $k$.

    \item The final basic property of Turing machines is the S-m-n theorem.
    This is in the notes.

    
\end{enumerate}

\section{Uncomputable Sets}
Does there exist an algorithm to solve every mathematical problem?
Turing showed it is theoretically impossible.

Theorem: there is a total function $f : \nat \to \nat$ that is not computable.

Proof? Cantor's theorem.
Review it every night before you go to sleep.

A key example of an incomputable function is
$$f(n) = 
\begin{dcases*}
    \phi_n(n) + 1 
    \text{if} \phi_n(n)\downarrow \\
    0 
    & \text{otherwise}
\end{dcases*}$$
$f$ is not computable because $f = \varphi_n$ and $f(n) = \varphi_n(n) + 1$,
but that would mean that $f(n) = \varphi_n(n) = \varphi_n(n) + 1$, which is a contradiction.

Let $H_0 = \{n \in \nat : \varphi_n(n) \downarrow\}$, and call this the halting set.
$H_0 \subseteq \nat$.

Theorem: $H_0$ is not computable.

A set of numbers is computable if its characteristic function $\chi_A()$
Assume $M_0$ computes $H_0$.
Then to compute our previous function $f$, on input $n$, first run $M_0(n)$.
If it outputs 0, then output 0.
If if outputs 1, call $\varphi_n(n)$, and output whatever it outputs.

Another uncomputable set is $H_1 = \{n \in \nat : \varphi_n(\varnothing)\downarrow\}$.
Idea: reduce the computation of $H_0$ to $H_1$.

\section{Rice's Theorem}
Rice's theorem says that no properties of computable functions can be determined computably from their indices.

\textbf{Definition:} Say $A \subseteq \nat$ is an \textit{index set} if for any $e \in A$, if $\varphi_e = \varphi_{e'}$, then $e' \in A$.
There are as many index sets as there are real numbers.

\textbf{Theorem (Rice):} No property of computable functions is computable.
Every computable index set is one of the following two sets: $\varnothing$ or $\nat$.
So any nontrivial collection of computable functions is incomputable.

\textit{Proof?} Assume $A \subseteq \nat$ is a nontrivial computable index set.
Reduce membership of $H_1$ to membership in $A$.
Transform $n$ to $h(n)$ such that $h(n) \in A \Iff n \in H_1$.
We will show this in full generality next time.

\section{Exercises}

\vspace{5mm}
\noindent \textbf{Exercise 1:} 
Is there an algorithm which takes a positive integer $n$ as input and outputs ``yes'' if there is a sequence of $n$ consecutive 7s in the decimal expansion of $\pi$ and ``no'' if there is not?

\vspace{2mm}
\noindent \textbf{Answer:} 


\vspace{5mm}
\noindent \textbf{Exercise 2:} Show that the set $\{n : \varphi_n(0)\downarrow\}$ is incomputable.


\vspace{2mm}
\noindent \textbf{Answer:} 
