The main topics of Metamathematics B are Gödel's Incompleteness Theorem and Computability Theory.

\section{Gödel's Incompleteness Theorem}
This is a theorem from 1931.
Gödel was trying to understand Hilbert's question of whether one could formalize all of mathematics into a single set of axioms and either deduce or refute every statement using that one set of axioms.
This was known as Hilbert's Program: the goal was to write down all the axioms of mathematics and prove that they are consistent (i.e. they do not lead to a contradiction).
However, in 1931, Gödel proved that this was in fact impossible;
there is no single set of axioms from which one can deduce all mathematical statements.

Roughly, Gödel showed the following:
Let ZFC denote the Zermelo-Fraenkel axioms for set theory (with the axiom of choice).
Then, if ZFC is consistent, then there exists a statement $\varphi$ which only quantifies over $\mathbb{N}$ such that ZFC does not prove or refute $\phi$.
In fact, one such $\varphi$ is ``ZFC consistent.'' (?)
In other words, ZFC cannot prove its own consistency.

If $T$ is computable extension of the ZFC axioms and $T$ is consistent, then $T$ does not prove $\con(T)$.

Suppose $\Sigma$ is a finite set.
$\Sigma^{< \infty}$ is the set of strings using alphabet $\Sigma$.
A subset of $\Sigma^{< \infty}$ is said to be computable, roughly, if
there is a computer program which, on input $s \in \Sigma^{< \infty}$, outputs $YES$ if $s \in T$ and $NO$ if $s \notin T$.
Morally, this means that the inclusion of $s$ in $T$ is decidable by a computer program.

\section{Turing Machines}
A Turing Machine is a model of a computer.
This model was defined to attempt to answer the questions
\begin{enumerate}
    \item Are there uncomputable sets?
    \item What is the structure of those sets?
    \item Do these uncomputable sets arise in mathematics?
\end{enumerate}

There are only countably many Fortran programs.
How many subsets of $\Sigma^{<\infty}$ are there?
There are uncountably many subsets of $\Sigma^{<\infty}$ (as long as $|\Sigma|< 1$).
So there are definitely incomputable sets.


\section{Primitive Recursive functions}
The \textit{primitive recursive functions} is the smallest family of functions from $\mathbb{N}^k \to \mathbb{N}$ contains
\begin{enumerate}
    \item Constant functions
    \item Successor function $n \mapsto n + 1$
    \item Projections $(x_1, ..., x_k) \mapsto x_i$
\end{enumerate}
Furthermore, if it closed under
\begin{enumerate}[label=\roman*]
    \item Composition
    \item Primitive recursion
\end{enumerate}

% TODO: turn this into a \definition
\textbf{Composition:}
If $f : \mathbb{N}^k \to \mathbb{N}$ and $g_1, ..., g_k$, where $g_i : \mathbb{N}^{k_i} \to \mathbb{N}$ are primitive recursive, so is $f(g_1, ..., g_k)$.

\textbf{Primitive recursion:}
If $g : \mathbb{N}^k \to \mathbb{N}$ and $f: \mathbb{N}^{k+2} \to \mathbb{N}$ are primitive recursive, then we can use $g$ as a base case and $f$ as the recursive step.
More formally, we can construct a new function 
$$h(y, x_1, ..., x_k) =
\begin{dcases*}
    g(x_1, ..., x_k) 
    & \text{if} y = 0 \\
    f(y, h(y-1, x_1, ..., x_k), x_1, ..., x_k) 
    & \text{otherwise}
\end{dcases*}
$$

We can enumerate all primitive recursive functions and then define a function $g$ such that
\begin{align*}
    g(0) &= f_{p_0}(0) + 1\\
    g(1) &= f_{p_0}(1) + f_{p_1}(1) + 1 \\
    &\vdots
\end{align*}
This diagonalization argument will give us a function that is not feasible to compute, but would be computable if we had all the time in the world.
What about primitive recursive functions allows this to work?

Herbrand-Gödel partial recursive functions are the smallest family of partial functions $\mathbb {N}^k \to \mathbb{N}$ containing (1), (2), and (3), and closed under (i) and (ii), and also such that (iii).

(iii) Minimization: If $f : \mathbb N ^{k+1} \to \mathbb N$ is partial recursive, then so is 
$$g(x_1, ..., x_k) = 
\min\{y \;|\; f(y, x_1, ..., x_k) = 0\}.$$
We call $g$ a minimization operation.
Note that no such $y$ is guaranteed to exist.
This could be either because there is no such $y$, or because $f$ runs forever (?)
The fact that some of these are not guaranteed to exist also eliminates the diagonalization argument because we cannot necessarily enumerate every partial recursive function.

A computable function is just a total partial computable function.

\section{Clerks}
Your clerk has access to
\begin{itemize}
    \item An infinite amount of paper
    \item An infinite amount of pencils
    \item Any other stationery items
\end{itemize}
The clerk can either write a symbol or erase a symbol.

Their mind can only enter finitely many ``states.''
These states determine how they are supposed to follow the rules.

You have a machine that has access to an infinite $\mathbb Z$-indexed tape.
The Turing Machine has a read head and a write head.
It can read and write symbols from a finite alphabet $\Sigma$.
The program formalized by this machine.

The transition function $t: Q \times \Sigma \to Q \times \Sigma \times \{L, R\}$ takes in a state $q$ and the current letter $s$ and 
$t(q, s) = (q', s', X)$, where $q'$ is the next state, $s'$ is the letter that should be written into the current cell, and $X$ determines whether to move the head of the machine either left or right.

In addition to this, Turing machines have two special states: $q_{start}$ and $q_{halt} \in Q$, which are the starting and halting state.
You can run this algorithm with any string.
We also have a special `blank' symbol, $\sqcup$.

%
Note: Turing machines can't starve.

$M$ computes the partial function $f:\Sigma^{<\infty} \to \Sigma^{<\infty}$ if when you run $M$ on a string $s$ if when $M$ halts, the output $f(s)$ is written on the tape.

A function $f: \Sigma^{<\infty} \to \Sigma^{<\infty}$ is computable if it can be computed by some Turing machine $M$.

Exercise: prove that the set of computable functions is the same as the primitive recursive functions.

Exercise: prove that every recursive function is a computable function (or the other way around)

$Succ : \{0, 1\}^{< \infty} \to  \{0, 1\}^{< \infty}$ is computable.

Exercise: find the error in Marks's notes.



For the next couple of weeks, we will be focusing on the basic properties of computable functions and the sets that can be computed by them.

