\documentclass{article}
\usepackage{graphicx} % Required for inserting images
\usepackage{amsmath}
\usepackage{amssymb}
\usepackage{amsthm}
\usepackage{enumitem}
\usepackage{verbatim}
\usepackage[left=25mm,right=25mm,top=25mm,bottom=25mm,paper=a4paper]{geometry}


\newcommand{\cL}{\mathcal{L}}
\newcommand{\cA}{\mathcal{A}}
\newcommand{\cAn}{\mathcal{A}_n}
\newcommand{\cB}{\mathcal{B}}
\newcommand{\cBn}{\mathcal{B}_n}
\newcommand{\cC}{\mathcal{C}}
\newcommand{\cF}{\mathcal{F}}
\newcommand{\cM}{\mathcal{M}}
\newcommand{\cN}{\mathcal{N}}
\newcommand{\cK}{\mathcal{K}}
\newcommand{\cS}{\mathcal{S}}
\newcommand{\UMi}{\bigcup_{i \in I}M_i}
\newcommand{\UcMi}{\bigcup_{i \in I}\cM_i}
\newcommand{\cG}{\mathcal{G}}
\newcommand{\cH}{\mathcal{H}}
\newcommand{\cR}{\mathcal{R}}
\newcommand{\real}{\mathbb{R}}
\newcommand{\nat}{\mathbb{N}}
\newcommand{\rat}{\mathbb{Q}}
\newcommand{\vphi}{\phi(v_1, v_2 ..., v_n)}
\newcommand{\cphi}{\phi(c_1, ..., c_n)}
\newcommand{\nmodels}{\nvDash}
\newcommand{\blank}{\; \;}
\newcommand{\diag}{\text{Diag}}
\newcommand{\diagel}{\text{Diag}_\text{el}}
\newcommand{\diagelm}{\text{Diag}_\text{el}(\cM)}
\newcommand{\Th}{\text{Th}}
\newcommand{\abar}{\overline{a}}
\newcommand{\bbar}{\overline{b}}
\newcommand{\cbar}{\overline{c}} 
\newcommand{\dbar}{\overline{d}} 
\newcommand{\ubar}{\overline{u}}
\newcommand{\vbar}{\overline{v}}
\newcommand{\wbar}{\overline{w}}
\newcommand{\xbar}{\overline{x}}
\newcommand{\ybar}{\overline{y}}
\newcommand{\ff}{\leftrightarrow}
\newcommand{\Iff}{\Leftrightarrow}
\newcommand{\rcf}{\mathsf{RCF}}
\newcommand{\tp}{\text{tp}}
\newcommand{\dom}{\text{dom}}
\newcommand{\ran}{\text{ran}}
\newcommand{\inverse}{^{-1}}
\newcommand\restrict[1]{\raisebox{-.5ex}{$|$}_{#1}}
\newcommand{\solution}{\noindent \textbf{Solution.}}
\newcommand{\lemma}{\noindent \textit{Lemma.}}
\newcommand{\acl}{\text{acl}}
\newcommand{\halts}{\downarrow}
\newcommand{\nhalts}{\uparrow}
% TODO: Add common math statements as macros


\title{MATH 225B Homework 2}
\author{Alejandro Sanchez Ocegueda}
\date{February 10, 2025}

\begin{document}

\maketitle

\section*{Exercise 1}
Is there an algorithm which takes a positive integer $n$ as input, and outputs ``yes'' if there is a sequence of $n$ consecutive 7s in the decimal expansion of $\pi$ and ``no'' if there is not?


\vspace{3mm}
\solution


\newpage
\section*{Exercise 2}
Suppose $A$ is an infinite c.e. set. Show that there is an infinite computable subset $B \subseteq A$.
[Hint: fix an enumeration $(A_s)_{s \in \nat}$ of $A$.
Then let $B = \{x : \exists s\;[x \in (A_{s + 1} \setminus A) \land x = \max (A_{s+1})]\}$ be the elements enumerated into $A$ that are greater than any previous elements that have been enumerated.]


\vspace{3mm}
\solution


\newpage
\section*{Exercise 3}
A c.e. set $A$ is computable iff it has a computable enumeration that has a computable modulus.

\vspace{3mm}
\solution



\newpage
\section*{Exercise 4}
Suppose $A, B \subseteq \nat$ are disjoint co-c.e.
Then show there is a computable set $C$ such that $A \subseteq C$ and $C \cap B = \varnothing$.


\vspace{3mm}
\solution

\newpage
\section*{Exercise 5}
Show that $TOT \nleq_m K$, 
so in the sense of many-to-one reducibility,
$TOT$ is strictly more complicated than $K$.
[Hint: Show that $\overline{K} \nleq_m TOT$.
Then conclude that if $TOT \leq_m K$, then $\overline{K} \leq_m K$, and so $\overline{K}$ would be c.e. Contradiction!]


\vspace{3mm}
\solution


\newpage
\section*{Exercise 6}
Let $\varphi_x\downarrow$ denote the Turing machine $x$ that halts on the empty input, and let $\langle \cdot, \cdot \rangle: \nat^2 \to \nat$ denote some computale bijection from $\nat^2 \to \nat$.
Let $K' = \{x : \varphi_x(0)\downarrow\}$, and $K'' = \{\langle x, y \rangle : \varphi_x(y)\downarrow\}.$
Show that $K \equiv_1 K' \equiv_1 K''$.

\vspace{3mm}
\solution


\newpage
\section*{Exercise 7}
Show that there is an $e$ so that $W_e = \{e\}$.
[Hint: Use Kleene's recursion theorem.]

\vspace{3mm}
\solution

\newpage
\section*{Exercise 8}
Show that there is no infinite c.e. set of minimal indices.

\vspace{3mm}
\solution

\newpage
\section*{Exercise 9}
Prove that $K$ is not an index set.

\vspace{3mm}
\solution

\end{document}