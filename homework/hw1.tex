\documentclass{article}
\usepackage{graphicx} % Required for inserting images
\usepackage{amsmath}
\usepackage{amssymb}
\usepackage{amsthm}
\usepackage{enumitem}
\usepackage{verbatim}
\usepackage[left=25mm,right=25mm,top=25mm,bottom=25mm,paper=a4paper]{geometry}


\newcommand{\cL}{\mathcal{L}}
\newcommand{\cA}{\mathcal{A}}
\newcommand{\cAn}{\mathcal{A}_n}
\newcommand{\cB}{\mathcal{B}}
\newcommand{\cBn}{\mathcal{B}_n}
\newcommand{\cC}{\mathcal{C}}
\newcommand{\cF}{\mathcal{F}}
\newcommand{\cM}{\mathcal{M}}
\newcommand{\cN}{\mathcal{N}}
\newcommand{\cK}{\mathcal{K}}
\newcommand{\cS}{\mathcal{S}}
\newcommand{\UMi}{\bigcup_{i \in I}M_i}
\newcommand{\UcMi}{\bigcup_{i \in I}\cM_i}
\newcommand{\cG}{\mathcal{G}}
\newcommand{\cH}{\mathcal{H}}
\newcommand{\cR}{\mathcal{R}}
\newcommand{\real}{\mathbb{R}}
\newcommand{\nat}{\mathbb{N}}
\newcommand{\rat}{\mathbb{Q}}
\newcommand{\vphi}{\phi(v_1, v_2 ..., v_n)}
\newcommand{\cphi}{\phi(c_1, ..., c_n)}
\newcommand{\nmodels}{\nvDash}
\newcommand{\blank}{\; \;}
\newcommand{\diag}{\text{Diag}}
\newcommand{\diagel}{\text{Diag}_\text{el}}
\newcommand{\diagelm}{\text{Diag}_\text{el}(\cM)}
\newcommand{\Th}{\text{Th}}
\newcommand{\abar}{\overline{a}}
\newcommand{\bbar}{\overline{b}}
\newcommand{\cbar}{\overline{c}} 
\newcommand{\dbar}{\overline{d}} 
\newcommand{\ubar}{\overline{u}}
\newcommand{\vbar}{\overline{v}}
\newcommand{\wbar}{\overline{w}}
\newcommand{\xbar}{\overline{x}}
\newcommand{\ybar}{\overline{y}}
\newcommand{\ff}{\leftrightarrow}
\newcommand{\Iff}{\Leftrightarrow}
\newcommand{\rcf}{\mathsf{RCF}}
\newcommand{\tp}{\text{tp}}
\newcommand{\dom}{\text{dom}}
\newcommand{\ran}{\text{ran}}
\newcommand{\inverse}{^{-1}}
\newcommand\restrict[1]{\raisebox{-.5ex}{$|$}_{#1}}
\newcommand{\solution}{\noindent \textbf{Solution.}}
\newcommand{\lemma}{\noindent \textit{Lemma.}}
\newcommand{\acl}{\text{acl}}
% TODO: Add common math statements as macros


\title{MATH 225B Homework 1}
\author{Alejandro Sanchez Ocegueda}
\date{January 31, 2025}

\begin{document}

\maketitle

\section*{Exercise 1}
Show that there is an algorithm for computing a function that is not primitive recursive.
[Hint: describe an algorithm for diagonalizing against all primitive recursive functions by defining a function $g$ where $g = f_n(n) + 1$ where $f_n$ is the $n$th primitive recursive function according to some listing of all such functions.]


\solution

\section*{Exercise 2}
\begin{enumerate}
    \item Show that if $f:\nat \to \nat$ is partial recursive, then there exists some $n$ so that $f(x_1, ..., x_n) \leq 2 \uparrow^n(\max_i(x_i)+3)$.
    \item Show that the function $2\uparrow^x x$ is not primitive recursive.
\end{enumerate}

\solution

\section*{Exercise 3}
Prove that every partial computable function from $\nat^k \to \nat$ is computable by a Turing machine.
[Hint: show that the functions computable by a Turing machine have all the properties 1-6 in the definition of partial computable functions.]

\solution

\section*{Exercise 4}
Show that $\{n: \varphi_n(0)\}$ is incomputable.

\solution


\end{document}