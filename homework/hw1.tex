\documentclass{article}
\usepackage{graphicx} % Required for inserting images
\usepackage{amsmath}
\usepackage{amssymb}
\usepackage{amsthm}
\usepackage{enumitem}
\usepackage{verbatim}
\usepackage[left=25mm,right=25mm,top=25mm,bottom=25mm,paper=a4paper]{geometry}


\newcommand{\cL}{\mathcal{L}}
\newcommand{\cA}{\mathcal{A}}
\newcommand{\cAn}{\mathcal{A}_n}
\newcommand{\cB}{\mathcal{B}}
\newcommand{\cBn}{\mathcal{B}_n}
\newcommand{\cC}{\mathcal{C}}
\newcommand{\cF}{\mathcal{F}}
\newcommand{\cM}{\mathcal{M}}
\newcommand{\cN}{\mathcal{N}}
\newcommand{\cK}{\mathcal{K}}
\newcommand{\cS}{\mathcal{S}}
\newcommand{\UMi}{\bigcup_{i \in I}M_i}
\newcommand{\UcMi}{\bigcup_{i \in I}\cM_i}
\newcommand{\cG}{\mathcal{G}}
\newcommand{\cH}{\mathcal{H}}
\newcommand{\cR}{\mathcal{R}}
\newcommand{\real}{\mathbb{R}}
\newcommand{\nat}{\mathbb{N}}
\newcommand{\rat}{\mathbb{Q}}
\newcommand{\vphi}{\phi(v_1, v_2 ..., v_n)}
\newcommand{\cphi}{\phi(c_1, ..., c_n)}
\newcommand{\nmodels}{\nvDash}
\newcommand{\blank}{\; \;}
\newcommand{\diag}{\text{Diag}}
\newcommand{\diagel}{\text{Diag}_\text{el}}
\newcommand{\diagelm}{\text{Diag}_\text{el}(\cM)}
\newcommand{\Th}{\text{Th}}
\newcommand{\abar}{\overline{a}}
\newcommand{\bbar}{\overline{b}}
\newcommand{\cbar}{\overline{c}} 
\newcommand{\dbar}{\overline{d}} 
\newcommand{\ubar}{\overline{u}}
\newcommand{\vbar}{\overline{v}}
\newcommand{\wbar}{\overline{w}}
\newcommand{\xbar}{\overline{x}}
\newcommand{\ybar}{\overline{y}}
\newcommand{\ff}{\leftrightarrow}
\newcommand{\Iff}{\Leftrightarrow}
\newcommand{\rcf}{\mathsf{RCF}}
\newcommand{\tp}{\text{tp}}
\newcommand{\dom}{\text{dom}}
\newcommand{\ran}{\text{ran}}
\newcommand{\inverse}{^{-1}}
\newcommand\restrict[1]{\raisebox{-.5ex}{$|$}_{#1}}
\newcommand{\solution}{\noindent \textbf{Solution.}}
\newcommand{\lemma}{\noindent \textit{Lemma.}}
\newcommand{\acl}{\text{acl}}
\newcommand{\halts}{\downarrow}
\newcommand{\nhalts}{\uparrow}
% TODO: Add common math statements as macros


\title{MATH 225B Homework 1}
\author{Alejandro Sanchez Ocegueda}
\date{January 31, 2025}

\begin{document}

\maketitle

\section*{Exercise 1}
Show that there is an algorithm for computing a function that is not primitive recursive.
[Hint: describe an algorithm for diagonalizing against all primitive recursive functions by defining a function $g$ where $g = f_n(n) + 1$ where $f_n$ is the $n$th primitive recursive function according to some listing of all such functions.]

\vspace{3mm}
\solution

Per the hint, suppose we have such a listing of the primitive recursive functions $f_0, f_1, f_2,$ and so on.
This enumeration can be achieved by use of a Gödel numbering.


We can then construct a table of these functions, where each row lists the values of $f_n$ at each natural number.
Below is the table:
\begin{table}[h!]
  \begin{center}
    \label{tab:table1}
    \begin{tabular}{|c|c|c|c} % <-- Alignments: 1st column left, 2nd middle and 3rd right, with vertical lines in between
    \hline
      $f_0(0)$ & $f_0(1)$ & $f_0(2)$ & \dots \\
      \hline
      $f_1(0)$ & $f_1(1)$ & $f_1(2)$ & \dots\\
      \hline
      $f_2(0)$ & $f_2(1)$ & $f_2(2)$ & \dots \\
      \hline
      \vdots & \vdots & \vdots & $\ddots$
    \end{tabular}
  \end{center}
\end{table}
Then, we can look at each diagonal entry (i.e. $f_n(n)$ for each $n \in \nat$), and define a function $g$ such that $g(n) = f_n(n) + 1$.
What this does is that it ensures that $g$ is different from every primitive recursive function at each step.

But could $g$ still be primitive recursive?
Suppose it is.
Then $g$ must appear at some point in the enumeration.
In other words, $g = f_{n^*}$ for some $n^* \in \nat$.
But by construction of $g$, we have that $g(n^*) = f_{n^*}(n^*) + 1$.
So this would mean that $f_{n^*}(n^*) = g(n^*) = f_{n^*}(n^*)+1$, a contradiction.

\section*{Exercise 2}
\begin{enumerate}
    \item Show that if $f:\nat^k \to \nat$ is partial recursive, then there exists some $n$ so that $f(x_1, ..., x_k) \leq 2 \uparrow^n(\max_i(x_i)+3)$.
    \item Show that the function $2\uparrow^x x$ is not primitive recursive.
\end{enumerate}

\vspace{3mm}
\solution

\begin{enumerate}
    \item We proceed by induction on the construction of the primitive recursive functions\footnote{I don't remember if this is the correct name for it, my apologies if I made a mistake here.}.

    \textbf{Base cases:}

    \begin{enumerate}
        \item \textit{Constant functions:} Suppose $f(x) = c$ for some constant $c$.
        Then we can easily choose $n$ such that $c \leq 2\uparrow^n 3$.
        \item \textit{Successor function:} Suppose $f(x) = x + 1$.
        Then we can choose $n = 1$, since $x + 1 \leq 2^{x + 3}$ for all $x$, meaning that $f(x) \leq 2 \uparrow^n (\max_i(x_i) + 3)$.
        \item \textit{Projection functions:} Suppose $f(x_1, ..., x_k) = x_j$ for some $j \in \{1, ..., k\}$.
        Then by the fact that $x_j \leq \max_i(x_i)$, we must have that
        $x_j \leq \max_i(x_i) + 3 \leq 2^{\max_i(x_i) + 3} = 2\uparrow^1(\max_i(x_i) + 3)$.
    \end{enumerate}
    Now we can proceed with the induction.

    \textbf{Induction:}

    \begin{enumerate}
        \item \textit{Composition:} Suppose $h: \nat ^k \to \nat$ is primitive recursive and $g_1, ..., g_k$ are also primitive recursive.
        Suppose further that $h(x_1, ..., x_k) \leq 2\uparrow^{n_0}(\max_i(x_i)+ 3)$ for some $n_0 \in \nat$, and each $g_i : \nat^{k_i} \to \nat$ satisfies $g_i(x_1, ..., x_{k_i}) \leq 2\uparrow^{n_i}(\max_i(x_i) + 3)$ for some $n_i \in \nat$.

        Then 

        \item \textit{Primitive recursion:} Suppose 
    \end{enumerate}
    
\end{enumerate}

\section*{Exercise 3}
Prove that every partial computable function from $\nat^k \to \nat$ is computable by a Turing machine.
[Hint: show that the functions computable by a Turing machine have all the properties 1-6 in the definition of partial computable functions.]

\vspace{3mm}
\solution

For this exercise, let us assume we are working with $\Sigma = \{0, 1\} \cup \{\sqcup\}$.

\begin{enumerate}
    \item It is easy to show that this Turing machine can compute the function $x \mapsto x$.
    Consider the program with states $Q = \{q_0\}$, where $q_0$ is the initial and halting state.
    Then, on input $x$ (in its binary representation), it will simply halt immediately.
    Thus the string that is left on the output tape is just $x$.
    % Let us define the transition function $t : \Sigma \times Q \to \Sigma \times Q \times \{L, R\}$ as follows:
    % \begin{align*}
    %     t(0, q_0) &= (0, q_1, R) \\
    %     t(1, q_0) &= (1, q_1, R) \\
    %     t(\sqcup, q_0) &= (\sqcup, q_1, R)
    % \end{align*}
    % Intuitively, this program will simply write back 
    \item To show that the Turing machine can compute the function $x \mapsto x+1$, we can use the same program given in Example 2.5 in the notes.
    Let $S = \{q_0, q_1, q_2\}$ and
    \item 
\end{enumerate}

\section*{Exercise 4}
Show that $\{n: \varphi_n(0) \downarrow\}$ is incomputable.

\vspace{3mm}
\solution

Consider the following function $g$:
$$g(n) = \begin{cases}
    \varphi_n(0) + 1
    & \text{if } \varphi_n(0)\downarrow\\
    0
    & \text{otherwise}
\end{cases}$$
We claim that this function is incomputable.
To see why, let us suppose that $g$ is computable.
Then $g = \varphi_m$ for some $m \in \nat$.
Now $\varphi_m$ is total, so it must be true that $g(0) = \varphi_m(0) + 1$, by definition of $g$.
However, this would contradict the fact that $g = \varphi_m$, since we would have that $\varphi_m(0) = g(0) = \varphi_{m}(0)+1$.

Now we claim that if $K_0$ were in fact computable, then we would be able to compute the function $g$ described above.
On input $n$, first compute whether $n \in K_0$.
If so, then $\varphi_n(0) \halts$, and thus we can use a universal Turing machine to simulate $\varphi_n$ and output $\varphi_n(0)$.
Otherwise, we know that $\varphi_n(0) \nhalts$, and thus we can simply output 0 as $g$ would do.

\end{document}