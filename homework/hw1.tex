\documentclass{article}
\usepackage{graphicx} % Required for inserting images
\usepackage{amsmath}
\usepackage{amssymb}
\usepackage{amsthm}
\usepackage{enumitem}
\usepackage{verbatim}
\usepackage[left=25mm,right=25mm,top=25mm,bottom=25mm,paper=a4paper]{geometry}


\newcommand{\cL}{\mathcal{L}}
\newcommand{\cA}{\mathcal{A}}
\newcommand{\cAn}{\mathcal{A}_n}
\newcommand{\cB}{\mathcal{B}}
\newcommand{\cBn}{\mathcal{B}_n}
\newcommand{\cC}{\mathcal{C}}
\newcommand{\cF}{\mathcal{F}}
\newcommand{\cM}{\mathcal{M}}
\newcommand{\cN}{\mathcal{N}}
\newcommand{\cK}{\mathcal{K}}
\newcommand{\cS}{\mathcal{S}}
\newcommand{\UMi}{\bigcup_{i \in I}M_i}
\newcommand{\UcMi}{\bigcup_{i \in I}\cM_i}
\newcommand{\cG}{\mathcal{G}}
\newcommand{\cH}{\mathcal{H}}
\newcommand{\cR}{\mathcal{R}}
\newcommand{\real}{\mathbb{R}}
\newcommand{\nat}{\mathbb{N}}
\newcommand{\rat}{\mathbb{Q}}
\newcommand{\vphi}{\phi(v_1, v_2 ..., v_n)}
\newcommand{\cphi}{\phi(c_1, ..., c_n)}
\newcommand{\nmodels}{\nvDash}
\newcommand{\blank}{\; \;}
\newcommand{\diag}{\text{Diag}}
\newcommand{\diagel}{\text{Diag}_\text{el}}
\newcommand{\diagelm}{\text{Diag}_\text{el}(\cM)}
\newcommand{\Th}{\text{Th}}
\newcommand{\abar}{\overline{a}}
\newcommand{\bbar}{\overline{b}}
\newcommand{\cbar}{\overline{c}} 
\newcommand{\dbar}{\overline{d}} 
\newcommand{\ubar}{\overline{u}}
\newcommand{\vbar}{\overline{v}}
\newcommand{\wbar}{\overline{w}}
\newcommand{\xbar}{\overline{x}}
\newcommand{\ybar}{\overline{y}}
\newcommand{\ff}{\leftrightarrow}
\newcommand{\Iff}{\Leftrightarrow}
\newcommand{\rcf}{\mathsf{RCF}}
\newcommand{\tp}{\text{tp}}
\newcommand{\dom}{\text{dom}}
\newcommand{\ran}{\text{ran}}
\newcommand{\inverse}{^{-1}}
\newcommand\restrict[1]{\raisebox{-.5ex}{$|$}_{#1}}
\newcommand{\solution}{\noindent \textbf{Solution.}}
\newcommand{\lemma}{\noindent \textit{Lemma.}}
\newcommand{\acl}{\text{acl}}
\newcommand{\halts}{\downarrow}
\newcommand{\nhalts}{\uparrow}
% TODO: Add common math statements as macros


\title{MATH 225B Homework 1}
\author{Alejandro Sanchez Ocegueda}
\date{January 31, 2025}

\begin{document}

\maketitle

\section*{Exercise 1}
Show that there is an algorithm for computing a function that is not primitive recursive.
[Hint: describe an algorithm for diagonalizing against all primitive recursive functions by defining a function $g$ where $g = f_n(n) + 1$ where $f_n$ is the $n$th primitive recursive function according to some listing of all such functions.]

\vspace{3mm}
\solution

Per the hint, suppose we have such a listing of the primitive recursive functions $f_0, f_1, f_2,$ and so on.
This can probably be done in a computable manner by using a Gödel numbering, assigning unique numbers for each of the constant functions, the successor function, and the projection functions, as well as the variable symbols (and the parentheses).

Once we have such an encoding, we can then construct a table of these functions, where each row lists the values of $f_n$ at each natural number.
Below is the table:
\begin{table}[h!]
  \begin{center}
    \label{tab:table1}
    \begin{tabular}{|c|c|c|c} % <-- Alignments: 1st column left, 2nd middle and 3rd right, with vertical lines in between
    \hline
      $f_0(0)$ & $f_0(1)$ & $f_0(2)$ & \dots \\
      \hline
      $f_1(0)$ & $f_1(1)$ & $f_1(2)$ & \dots\\
      \hline
      $f_2(0)$ & $f_2(1)$ & $f_2(2)$ & \dots \\
      \hline
      \vdots & \vdots & \vdots & $\ddots$
    \end{tabular}
  \end{center}
\end{table}
Then, we can look at each diagonal entry (i.e. $f_n(n)$ for each $n \in \nat$), and define a function $g$ such that $g(n) = f_n(n) + 1$.
What this does is that it ensures that $g$ is different from every primitive recursive function at each step.

But could $g$ still be primitive recursive?
Suppose it is.
Then $g$ must appear at some point in the enumeration.
In other words, $g = f_{n^*}$ for some $n^* \in \nat$.
But by construction of $g$, we have that $g(n^*) = f_{n^*}(n^*) + 1$.
So this would mean that $f_{n^*}(n^*) = g(n^*) = f_{n^*}(n^*)+1$, a contradiction.

\newpage
\section*{Exercise 2}
\begin{enumerate}
    \item Show that if $f:\nat^k \to \nat$ is primitive recursive, then there exists some $n$ so that $f(x_1, ..., x_k) \leq 2 \uparrow^n(\max_i(x_i)+3)$.
    \item Show that the function $2\uparrow^x x$ is not primitive recursive.
\end{enumerate}

\vspace{3mm}
\solution

\begin{enumerate}
    \item We proceed by induction on the construction of the primitive recursive functions\footnote{I don't remember if this is the correct name for it, my apologies if I made a mistake here.}.

    \textbf{Base cases:}

    \begin{enumerate}
        \item \textit{Constant functions:} Suppose $f(x) = c$ for some constant $c$.
        Then we can easily choose $n$ such that $c \leq 2\uparrow^n 3 \leq 2\uparrow^n (x + 3)$.
        For instance, we can choose $n = c$ (it's a bit overkill, but it will do).
        \item \textit{Successor function:} Suppose $f(x) = x + 1$.
        Then we can choose $n = 1$, since $x + 1 \leq 2^{x + 3}$ for all $x$, meaning that $f(x) \leq 2 \uparrow^n (\max_i(x_i) + 3)$.
        \item \textit{Projection functions:} Suppose $f(x_1, ..., x_k) = x_j$ for some $j \in \{1, ..., k\}$.
        Then by the fact that $x_j \leq \max_i(x_i)$, we must have that
        $x_j \leq \max_i(x_i) + 3 \leq 2^{\max_i(x_i) + 3} = 2\uparrow^1(\max_i(x_i) + 3)$.
        We can again choose $n = 1$.
    \end{enumerate}
    Now we can proceed with the induction.

    \textbf{Induction:}

    \begin{enumerate}
        \item \textit{Composition:} Suppose $h: \nat ^k \to \nat$ is primitive recursive and $g_1, ..., g_k$ are also primitive recursive.
        Suppose further that $h(x_1, ..., x_k) \leq 2\uparrow^{n_0}(\max_i(x_i)+ 3)$ for some $n_0 \in \nat$, and each $g_i : \nat^{k_i} \to \nat$ satisfies $g_i(x_1, ..., x_{k_i}) \leq 2\uparrow^{n_i}(\max_i(x_i) + 3)$ for some $n_i \in \nat$.

        Let $n^* = \max(n_1, ..., n_k)$.
        Then for $i \in \{1, ..., k\}$, we have that
        $g_i(x_1, ..., x_{k_i}) \leq 2\uparrow^{n^*}(\max_j(x_j) + 3)$.

        Furthermore, this means that
        \begin{align*}
        h(g_1, .., g_k) 
        &\leq 2\uparrow^{n_0} (\max(g_1, ..., g)_k) + 3)\\
        &\leq 2\uparrow^{n_0} ((2\uparrow^{n^*}(\max_j(x_j) + 3)) + 3)
        % &\leq 2\uparrow^{n_0 + n^* + 1}(\max_j(x_j) + 3)
        \end{align*}
        Since $2\uparrow^{n^*}(\max_j(x_j) + 3) + 3 \leq 2\uparrow^{n^* + 1} (\max_j(x_j) + 3)$, we can then simplify this further to
        \begin{align*}
        h(g_1, ..., g_k) &\leq 2\uparrow^{n_0}(2\uparrow^{n^* + 1} (\max_j(x_j) + 3))\\
            &\leq 2\uparrow^{n_0 + n^* + 1} (\max_j(x_j)+3)
        \end{align*}
        So we can let $n = n_0 + n^* + 1$, and we are done.

        \item \textit{Primitive recursion:} Suppose $g: \nat^k \to \nat$ and $h:\nat^{k+2} \to \nat$ are primitive recursive and
        \begin{align*}
            g(x_1, ..., x_k) &\leq 2\uparrow^{n_g}(\max_i(x_i) + 3)\\
            \land h(y, z, x_1, ..., x_k) &\leq 2 \uparrow^{n_h}(\max(y, z,\max_i(x_i)) + 3)
        \end{align*}
        Then by the definition of primitive recursion, the function 
        $$f(y, x_1, ..., x_k) = \begin{cases}
            g(x_1, ..., x_k)
            & \text{if }y=0
            \\
            h(y-1, f(y-1,x_1,..., x_k), x_1,..., x_k)
            & \text{if }y > 0
        \end{cases}$$
        must be primitive recursive too.
        We now prove by induction on $y$ that for all $y \geq 0$,
        $f(y, x_1, ..., x_k) \leq 2\uparrow^n (\max(y, \max_i(x_i)) + 3)$ for some $n \in \nat$.
        
        \textbf{Base Case:}
        If $y = 0$, then $f(y, x_1, ..., x_k) = g(x_1, ..., x_k) \leq g(x_1, ..., x_k) \leq 2\uparrow^{n_g}(\max_i(x_i) + 3) \leq 2\uparrow^{n_g}(\max(0, \max_i(x_i)) + 3)$.
        So the base case holds for $n = n_g$.

        \textbf{Inductive Step:} Suppose $f(y, x_1, ..., x_k) \leq 2\uparrow^n (\max(y, \max_i(x_i)) + 3)$ for $y > 0$.
        Then we have that
        \begin{align*}
        f(y+1, x_1, ..., x_k) &= h(y, f(y, x_1, ..., x_k), x_1, ..., x_k) \\
            &\leq 2 \uparrow^{n_h}(\max(y, z,\max_i(x_i)) + 3) \\
            &= 2\uparrow^{n_h} (\max(y, f(y, x_1, ..., x_k), \max_i(x_i)) + 3) \\
            &\leq 2\uparrow^{n_h}(\max(y, {2\uparrow^n (\max(y, \max_i(x_i)) + 3)}, \max_i(x_i)) + 3)\\ 
            &\leq 2\uparrow^{n_h}(\max(y+1, {2\uparrow^n (\max(y+1, \max_i(x_i)) + 3)}, \max_i(x_i)) + 3) \\
            &\leq 2\uparrow^{n_h}({2\uparrow^n (\max(y+1, \max_i(x_i)) + 3)}) + 3) \\
            &\leq 2\uparrow^{n+1}(\max(y, \max_i(x_i)) +3)
        \end{align*}
        So we set $n = \max(n_h, n_g) + 1$.
    \end{enumerate}
    Thus we conclude that for any primitive recursive function $f: \nat^k \to \nat$, $f(x_1, ..., x_k) \leq 2\uparrow^n(\max_i(x_i) + 3)$ for some $n \in \nat$.

    \item We can use our results from the previous subpart to show this.
    Suppose $2\uparrow^x x$ is primitive recursive for the sake of contradiction.
    Then there should exist some $n \in \nat$ such that
    $$2\uparrow^x x \leq 2 \uparrow^n (x + 3)$$
    But if we pick $x = n + 1$, we find that
    \begin{align*}
        2\uparrow^{n+1}(n+1) 
        &= 2\uparrow^n(2\uparrow^{n+1}n) \\
        &> 2\uparrow^n (n + 4) \;\;\;\text{ for any sufficiently large }n
    \end{align*}
    This is a contradiction to the last subpart.
    Therefore, we conclude that $2\uparrow^x x$ is \textit{not} primitive recursive, as desired.
\end{enumerate}

\newpage
\section*{Exercise 3}
Prove that every partial computable function from $\nat^k \to \nat$ is computable by a Turing machine.
[Hint: show that the functions computable by a Turing machine have all the properties 1-6 in the definition of partial computable functions.]

\vspace{3mm}
\solution

For this exercise, let us assume we are working with $\Sigma = \{0, 1\} \cup \{\sqcup\}$.

\begin{enumerate}
    \item It is easy to show that this Turing machine can compute the function $x \mapsto x$.
    Consider the program with states $Q = \{q_0\}$, where $q_0$ is the initial and halting state.
    Then, on input $x$ (in its binary representation), it will simply halt immediately.
    Thus the string that is left on the output tape is just $x$.
    \item To show that the Turing machine can compute the function $x \mapsto x+1$, we can a similar program to the one given in Example 2.5 in the notes.
    Let $Q = \{q_0, q_1, q_2, q_3\}$ and define the transition function $t : \Sigma \times Q \to \Sigma \times Q \times \{L, R\}$ as
    \begin{align*}
        t(0, q_0) &= (0, q_0, R) \\
        t(1, q_0) &= (1, q_0, R) \\
        t(\sqcup, q_0) &= (\sqcup, q_1, L) \\
        t(0, q_1) &= (1, q_2, L) \\
        t(1, q_1) &= (0, q_1, L) \\
        t(\sqcup, q_1) &= (1, q_2, L) \\
        t(0, q_2) &= (0, q_2, L) \\
        t(1, q_2) &= (1, q_2, L) \\
        t(\sqcup, q_2) &= (\sqcup, q_3, R)
    \end{align*}
    Note that we are assuming that the tape head starts on the most significant bit.
    
    \item To show that a Turing machine can compute the projection function $f(x_1, ..., x_k) = x_i$, we can use the program specified by $Q = \{q_1, ..., q_{k+3}\}$. 
    Note that we assume that the input is encoded as $x_1 \sqcup x_2 \sqcup ... \sqcup x_k$, and that the tape head is positioned at the most significant bit of $x_1$.
    We designate $q_1$ as the initial state and $q_k$ as the halting state. 
    Transition function $t: \Sigma \times Q \to \Sigma \times Q \times \{L, R\}$ that works as described below.
    For $j < i$:
    \begin{align*}
        t(0, q_j) &= (\sqcup, q_j, R)\\
        t(1, q_j) &= (\sqcup, q_j, R)\\
        t(\sqcup, q_j) &= (\sqcup, q_{j+1}, R) \\
    \end{align*}
    For $i$:
    \begin{align*}
        t(0, q_i) &= (0, q_i, R)\\
        t(1, q_i) &= (1, q_i, R)\\
        t(\sqcup, q_i) &= (\sqcup, q_{i+1}, R) \\
    \end{align*}
    For $i < j \leq k$:
    \begin{align*}
        t(0, q_j) &= (\sqcup, q_j, R)\\
        t(1, q_j) &= (\sqcup, q_j, R)\\
        t(\sqcup, q_j) &= (\sqcup, q_{j+1}, R) \\
    \end{align*}
    Next, we have
    \begin{align*}
        t(0, q_{k+1}) &= (0, q_{k+2}, L)\\
        t(1, q_{k+1}) &= (1, q_{k+2}, L)\\
        t(\sqcup, q_{k+1}) &= (\sqcup, q_{k+1}, L) \\
    \end{align*}
    and finally, we have that
    \begin{align*}
        t(0, q_{k+2}) &= (0, q_{k+2}, L)\\
        t(1, q_{k+2}) &= (1, q_{k+2}, L)\\
        t(\sqcup, q_{k+2}) &= (\sqcup, q_{k+3}, R) \\
    \end{align*}
    Intuitively, this program starts at $x_1$, then sweeps to the right, deleting all entries except for $x_i$.
    Then, it goes back and resets the tape head to point to the most significant bit of $x_i$.
    
    \item Assume $h: \nat^k \to \nat$ is computable and partial recursive, and that $g_1, ..., g_k$ are computable and partial recursive. 
    Then we can show that $h(g_1, ..., g_k)$ is also computable and partial recursive.
    Let $M_h$ denote the Turing machine that computes function $h$, and $M_{g_i}$ similarly be the Turing machines that compute $g_i$ for each $i \in \{1, ..., k\}$.
    The idea is that we can construct a machine $M$ that computes the composition of these functions.
    To do this, the Turing machine first simulates each $g_i$ on their input and writes the output onto the tape in a reserved region.
    It can track which machine it is currently simulating by careful use of the states of the machine.
    Once it is done writing the output to the tape, it then simulates $h$, using the previous output of the $g_i$ as input.
    
    Note that if $h$ or any of the $g_i$ are undefined, then $h(g_1, ..., g_k)$ is also undefined. 
    This is reflected in our machine $M$.
    Since, by assumption, each $M_{g_i}$ and $M_h$ halt if and only if their corresponding function is defined, then $M$ will also only halt if all of these machines halt.
    Thus $M$ halts if and only if $h(g_1, ..., g_k)$ is defined.

    \item Suppose $g : \nat^k \to \nat$ and $h: \nat^{k+2}\to \nat$ are partial recursive and computable by machines $M_g$ and $M_h$, respectively.
    Then the function $f: \nat^{k+1} \to \nat$ defined by
    $$f(y, x_1, ..., x_k) = \begin{cases}
            g(x_1, ..., x_k)
            & \text{if }y=0
            \\
            h(y-1, f(y-1,x_1,..., x_k), x_1,..., x_k)
            & \text{if }y > 0
        \end{cases}$$
        is also computable and partial recursive.
    We proceed by induction on $y$.

    \textbf{Base case:}
    If $y= 0$, then $M_f$ can simply simulate $M_g(x_1, ..., x_k)$ to get the desired behavior.
    This works because 
    \begin{align*}
        f(0, x_1, ..., x_k) &= g(x_1, ..., x_k)\\
        &= M_g(x_1, ..., x_k)
    \end{align*}
    So $M_f$ would correctly compute $f(y, x_1, ..., x_k)$ for $y=0$.
    % If $y > 0$, then we can start at $y=0$, also simulating $M_g(x_1, ..., x_k)$.
    % After that, we can use the output of $M_g(x_1, ..., x_k)$ as input to $M_h$ to compute 
    % $$M_h(1, f(y-1, x_1, ..., x_k), x_1, ..., x_k)$$
    
    \textbf{Inductive step:}
    For $y > 0$, assume inductively that $M_f$ correctly computes $f(y, x_1, ..., x_k)$.
    Then for $y + 1$, we can simply observe that
    \begin{align*}
        f(y, x_1, ..., x_k) &= h(y, f(y, x_1, ..., x_k), x_1, ..., x_k) \\
        &= h(y, M_f(y, x_1, ..., x_k), x_1, ..., x_k ) \\
        &= M_h(y, M_f(y, x_1, ..., x_k), x_1, ..., x_k)
    \end{align*}
    Thus, by our inductive assumption and our other assumption that $h$ is computable by $M_h$, we find that $M_f$ also computes $f(y, x_1, ..., x_k)$ correctly for $y > 0$.

    \item Suppose $f:\nat^{k+1} \to \nat$ is partial recursive and computable by a Turing machine $M_f$.
    Then the function $g: \nat^k \to \nat$ where $g(x_1, ..., x_k)$ is equal to the least such $y$ such that $f(y, x_1, ..., x_k) = 0$ if such a $y$ exists and for all $y' < y$ we have that $f(y', x_1, ..., x_k)$ is defined.
    Otherwise, $g(x_1, ..., x_k)$ is undefined.
    Furthermore, $g$ is computable by a Turing machine $M_g$.

    $M_g$ behaves as follows:
    It begins with $y = 0$, and it simulates $M_f(0, x_1, ..., x_k)$.
    If $M_f(0, x_1, ..., x_k) = 0$, then $M_g$ halts and outputs 0.
    Otherwise, it moves on to $y=1$: $M_g$ simulates $M_f(1, x_1, ..., x_k)$.
    Again, if $M_f(1, x_1, ..., x_k) = 0$, then $M_g$ halts and outputs $1$.
    Otherwise, it tries again with $y =2$.
    This procedure is repeated for each subsequent value of $y$.

    If some $y$ exists such that $M_f(y, x_1, ...,x_k) = 0$, then $M_g$ will halt and output $y$, provided that for $y'< y$, $M_g(y', x_1, ..., x_k)$ halts (which happens if and only if $f(y', x_1, ..., x_k)$ is defined).
    Otherwise, $M_g$ will run forever.
    This is in line with the requirements of the function $g$ described above.
\end{enumerate}
This concludes the proof that every function $f: \nat^k \to \nat$ is computable by a Turing machine.

\newpage
\section*{Exercise 4}
Show that $\{n: \varphi_n(0) \downarrow\}$ is incomputable.

\vspace{3mm}
\solution

Consider the following function $g$:
$$g(n) = \begin{cases}
    \varphi_n(0) + 1
    & \text{if } \varphi_n(0)\downarrow\\
    0
    & \text{otherwise}
\end{cases}$$
We claim that this function is incomputable.
To see why, let us suppose that $g$ is computable.
Then $g = \varphi_m$ for some $m \in \nat$.
Now $\varphi_m$ is total, so it must be true that $g(0) = \varphi_m(0) + 1$, by definition of $g$.
However, this would contradict the fact that $g = \varphi_m$, since we would have that $\varphi_m(0) = g(0) = \varphi_{m}(0)+1$.

Now we claim that if $K_0$ were in fact computable, then we would be able to compute the function $g$ described above.
On input $n$, first compute whether $n \in K_0$.
If so, then $\varphi_n(0) \halts$, and thus we can use a universal Turing machine to simulate $\varphi_n$ and output $\varphi_n(0)$.
Otherwise, we know that $\varphi_n(0) \nhalts$, and thus we can simply output 0 as $g$ would do.

\end{document}